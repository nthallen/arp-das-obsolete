\parindent 0pt
\parskip 12pt
\line{\hfil \bf System Controller \hfil \llap{\rm Mar. 12, 1990}}
\centerline{Functional Description}
\vskip .5in
The System Controller provides all the functions previously located on
the PC/ICC, SIC and ICC Buffer cards while eliminating the subbus
cable.  This board in combination with the small Ampro 286 motherboard
makes it possible to locate the computer physically inside the ICC.

{\obeylines
{\bf Features:}
{\parskip 0pt\leftskip 12pt
   SUBBUS address/data for read/write with acknowledge
   Command Strobe
   CMDENBL/failsafe timer/system reset
   8 bit input port for reading front panel switches
   8K $\times$ 8 static RAM with battery backup
   Power failure NMI
   LED status indicators for system integrity functions
}
{\bf Options:}
{\parskip 0pt\leftskip 12pt
   IRQ selection
   NMI enable
   System Reset enable
}}

\vskip 24pt
\hbox to \hsize{\hfil
\vbox{\offinterlineskip
\halign{\strut\vrule#\quad& \hfil#\hfil& \quad\vrule\quad\hfil#\hfil&
        \quad\vrule\quad#\hfil\quad\vrule\cr
  \multispan4\hrulefill\cr
  & \hidewidth {\bf Address}\hidewidth & {\bf R/W} &
       {\bf Description}\cr
  \multispan4\hrulefill\cr
  & 308--309& R/W& 8255 Port A\cr
  \multispan4\hrulefill\cr
  & 30A--30B& R/W& 8255 Port B\cr
  \multispan4\hrulefill\cr
  & 30C--30D& R/W& 8255 Port C\cr
  \multispan4\hrulefill\cr
  & 30E--30F& R/W& 8255 Control Port\cr
  \multispan4\hrulefill\cr
  & 310& W& Produces 8255 reset\cr
  \multispan4\hrulefill\cr
  & 311& W& Disarms the reset circuit and resets\cr
  &    && CMDENBL, LOWPOWER and NMI Enable\cr
  \multispan4\hrulefill\cr
  & 318& W& Sets CMDENBL to value of D0\cr
  \multispan4\hrulefill\cr
  & 319& W& Produces TICK for failsafe timing\cr
  \multispan4\hrulefill\cr
  & 31A& W& Sets non-volatile RAM address latch\cr
  \multispan4\hrulefill\cr
  & 31C& R& General input port\cr
  &    & W& NMI Enable Register\cr
  \multispan4\hrulefill\cr
  & 31D& R/W& Non-volatile RAM data\cr
  \multispan4\hrulefill\cr
}} \hfil}

\filbreak\vskip 24pt
\leftline{\bf SUBBUS Operation}

The SUBBUS interface is controlled through two 8255 Programmable
Peripheral Interface chips.  One 8255 is located on the low half of
the AT data bus, and the other is located on the high half of the bus. 
Being programmable, these chips require some software configuration
before use.  The SUBBUS configuration calls for the following:

\filbreak\vskip 24pt
\vbox{\offinterlineskip
\halign{\strut\vrule#& \quad\hfil#\hfil\quad& \vrule\quad#& #\hfil&
		\vrule#\quad& #\hfil\quad& \vrule#\cr
    \multispan7\hrulefill\cr
    &  {\bf port}&& \omit{\bf \hfil mode\hfil}&&
         \omit\hfil{\bf use}\hfil&\cr
    \multispan7\hrulefill\cr
    & A&& Mode 2 (Strobed Bidirectional Bus I/O)&& ICC Data Bus&\cr
    \multispan7\hrulefill\cr
    & B&& Mode 0 Output&& ICC Address Bus&\cr
    \multispan7\hrulefill\cr
    & C3--C7&& Special purpose for Mode 2&& C5: IBF$_A$&\cr
    \multispan7\hrulefill\cr
    & C0--C2&& Low byte: Mode 0, Output&&  C0: EXPRD&\cr
    &       &&                         &&  C1: CMDSTROBE&\cr
    &       &&                         &&  C2: EXPWR&\cr
    &       && High byte: Mode 0, Input&&  C0: EXPACK/&\cr
    &       &&                         &&  C1: CMDENBL/&\cr
    &       &&                         &&  C2: LOWPOWER&\cr
    \multispan7\hrulefill\cr
}}

The first step to initializing the 8255s is to issue a hardware reset�
by writing to I/O port 310H.  Then write the control word C1C0H to the�
control ports at address 30EH.  The SUBBUS is now fully configured.

The read and write control signals on the ICC backplane are under full�
software control.  In order to issue a read or a write instruction,�
the appropriate control signal is asserted through port C of the 8255�
on the low half of the data bus.  The cycles are defined as follows:

{\obeylines\vskip 12pt\parskip 0pt\parindent 24pt
    {\noindent\bf Write Cycle:}
      Write the SUBBUS address to port B.
      Write the data value to port A.
      Set EXPWR high.
      Wait an appropriate length of time.
      Check for EXPACK/ by reading the high byte of port C.
      Reset EXPWR.
\vskip 12pt
    {\noindent\bf Read Cycle:}
      Write the SUBBUS address to port B.
      Set EXPRD high.
      Wait an appropriate length of time.
      Reset EXPRD.
      Check IBF (C5 on the low byte of port C.)
      Read data from port A.
}

The write cycle is considered successful if C0 on the high half of the�
bus goes low during the write cycle.  The read cycle is successful if�
IBF (C5 on the low half) is high after the read cycle.

Since most of the bits of port C serve special purposes, it is advised�
that they be manipulated using the bit set/reset function of the 8255. �
The bit set/reset command format is a binary number of the form�
$0000bbbv_2$ where $bbb$ is the bit number, and $v$ is the new value�
for that bit.  This command should be written to the control port for�
the appropriate 8255, in this case the one on the low half of the bus�
at address 30EH.

\filbreak\vskip 24pt
\leftline{\bf Command Strobe (CMDSTROBE)}

The command strobe is manipulated in the same manner as the SUBBUS�
read and write signals.  To assert CMDSTROBE, write the byte value 3�
to I/O port 30EH.  To reset CMDSTROBE, write the byte value 2 to 30EH.

\filbreak\vskip 24pt
\leftline{\bf CMDENBL, failsafe timer and system reset}

The PC/ICC architecture specifies that all discrete commands eminating�
from the ICC be qualified not only by the command strobe but by a�
command enable signal.  The purpose of this signal is to lessen the�
possibility that the computer will issue erroneous commands while it�
is in the process of crashing.

How do we know if the computer is qualified to issue commands?  The�
system controller board features a failsafe timer which is capable of�
resetting the computer in the event of a software or hardware�
malfunction which interrupts normal operation.  The timer is activated�
by writing to the TICK address (319H).  Once the timer is activated, a�
TICK must be written every second or the board will issue a hardware�
reset.  The CMDENBL signal may be set or reset under software control�
by writing 1 or 0 to address 318H, but it is enabled onto the�
backplane only when the failsafe timer is activated.  The timer may be�
deactivated by writing to address 311H.  This not only disarms the�
failsafe timer but also resets CMDENBL so that it must be explicitly�
re\"enabled if the timer is reactivated.

For diagnostic purposes, the CMDENBL/ signal which appears on the ICC�
backplane may be read on bit C1 of the high 8255.

\filbreak\vskip24pt
\leftline{\bf 8-bit Input Port}

An 8-bit input port can be read at I/O address 31CH.  All bits may be�
application dependent and may be connected to front panel switches for�
operational mode selection.

\filbreak\vskip24pt
\leftline{\bf Non-volatile RAM}

The 8K$\times$8 non-volatile RAM is accessed by first writing the�
appropriate address (from 0 to 1FFFH) to the RAM Address Latch at I/O�
address 31AH.  The selected byte is then read or written from I/O�
address 31DH.  Note that only byte transfers are supported.

The contents of the non-volatile RAM will be subject to conventions
which arise over the course of application development.  This process
will be facilitated if we specify some basic groundrules at this time.
In general, the byte at address 0 will contain the current mode of
operation of the system.  This information is most useful after a
power outage or a system reboot for determining how to start up
again.  Other data bytes may have permanent significance, but in
general, the meaning of all of the other data bytes depends on the
value of the first status byte.  In particular, several high status
values will be allotted to diagnostic states and will indicate that
the rest of the RAM should contain either a test pattern or data
pertaining to some other diagnostic test.


\filbreak\vskip 24pt
\leftline{\bf Status Byte Definitions:}
\vskip 12pt
\vbox{\offinterlineskip\parskip 0pt\parindent 0pt
\halign{\strut\vrule#& \quad\hfil#\hfil\quad& \vrule\quad#&
		\vtop{\hsize 3in \strut #\strut} \hfil&
		\vrule#\cr
    \multispan5\hrulefill\cr
    &  {\bf value}&& \omit{\bf \hfil meaning\hfil}&\cr
    \multispan5\hrulefill\cr
    &  F7 && System reset diagnostic in progress&\cr
    \multispan5\hrulefill\cr
    &  F8 && Power fail diagnostic in progress&\cr
    \multispan5\hrulefill\cr
    & F9--FF && Test Patterns: Remainder of RAM is filled with
		the specified value or pattern:&\cr
    \multispan5\hrulefill\cr
    &  F9 && Test Pattern 0: $f(A) = (A < 256) ? A : 511 - A;$\break
		where $A$ is the RAM address&\cr
    \multispan5\hrulefill\cr
    &  FA && Test Pattern 1: 0&\cr
    \multispan5\hrulefill\cr
    &  FB && Test Pattern 2: FF&\cr
    \multispan5\hrulefill\cr
    &  FC && Test Pattern 3: 55&\cr
    \multispan5\hrulefill\cr
    &  FD && Test Pattern 4: AA&\cr
    \multispan5\hrulefill\cr
    &  FE && Test Pattern 5: 0F&\cr
    \multispan5\hrulefill\cr
    &  FF && Test Pattern 6: F0&\cr
    \multispan5\hrulefill\cr
}}

\filbreak\vskip24pt
\leftline{\bf Power failure NMI}

The System Integrity Control chip (U27 MAX691) is capable of producing�
a non-maskable interrupt (NMI) if it detects that the power to the
computer is failing.  If the Power Fail Input (PFI) on pin 9 of U27
drops below 3.6 Volts, the Power Fail Output (PFO) will be asserted.
This will latch the LOWPOWER signal, which may be read as bit C2 on
the high 8255.  LOWPOWER is reset on a system reset, DISARM command,
or when any value is written to the NMI Enable Register.

The NMI can be enabled by writing a value of 1 to the NMI Enable
Register at address 31CH.  It may be disabled again by writing a 0
value.  NMI is also disabled on power up, system reset or by
the DISARM command.

Since NMI is used for other purposes in the PC/AT architecture, the
NMI routine must first ascertain whether or not the System Controller
produced the interrupt.  If LOWPOWER is asserted, we can assume the
that is the cause of the NMI.  Otherwise, the interrupt should be
handed on to the next NMI routine.

A LOWPOER NMI routine might attempt to write critical information�
to the non-volatile RAM and otherwise secure the system before a�
system reset.

\filbreak\vskip24pt
\leftline{\bf LED Status Indicators}

The board contains provisions for three LED status indicators for�
monitoring the system integrity functions.  None of these lights�
should be displayed if the Vcc power is off.  The first light is�
displayed if the watchdog timer has timed out.  This will be true�
before software has taken control of the board and started issuing the�
periodic TICK.  The second light is normally on, but will turn off if�
the non-volatile RAM is being powered by the battery.  The third LED�
turns on if Vcc drops below 4.65 Volts.  Since it is pulled up to Vcc,�
it will only display until Vcc drops below .6 Volts, but if Vcc were�
to hang out between .6 and 4.65 Volts, this LED would be on.

\filbreak\vskip24pt
\leftline{\bf Diagnostics}

Each of these systems must be tested and the test programs included in�
the diagnostic suite.  Descriptions of the diagnostics will follow�
here at a later date.

\end
